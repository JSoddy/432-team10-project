\documentclass[11pt]{article}
\usepackage[margin=1in]{geometry}

%opening
\title{CSCI 432, Group Project Part 2}
\author{Group 10}

\begin{document}

\maketitle

For this part of the project, we were instructed to choose three
algorithms of interest to us. The first we chose is the SplitMEM
Algorithm. This algorithm is designed to take multiple genomes
from a single species and create one "pan­genome" that is
representative of the species as a whole. Next, we chose to look
at "Data Structures with Unpredictable Timing," from Bethea and
Reiter 2009. In this paper they explore the idea of creating data
structures with unpredictable timing, which retain efficiency, in
the interest of preventing denial­of­service attacks. Finally, we
look at quantum computation and factoring. Quantum computation,
being based on quantum physics, is fundamentally different from
traditional computing and presents possible better solutions to
such problems as generating random numbers and factoring numbers.
Below are further explanations of each topic:

\section{Algorithm: SplitMEM}
The area of genetic sequencing has been enjoying a period of exponential
growth for several decades now. Breakthroughs in chemistry, as well as 
significant improvements in sequencing equipment, have led to a rapid 
decline in the cost of sequencing an organism's genetic data. At the 
same time, the increasing power of computing machinery and the decreasing
cost of storage have encouraged the ever-greater accumulation of data in 
a large variety of areas. These two pressures have made, and will 
continue to make, the sequencing of DNA from large numbers of organisms 
more accessible. Already, there are projects in various stages of 
completion which aim to sequence hundreds, or even thousands, of genomes 
from a single species or clade\cite{Haussler08}\cite{1000genomes}\cite{1001genomes}\cite{100KProject}. 
While these developments are beneficial 
to the fields of biology and genomics, they do present new challenges 
and highlight shortcomings in the existing models used to represent 
genetic data. The current standard in genomics is to use the genome of a 
single organism as a reference for its species. However, this can lead to
biases such as a preference for recording gene sequences in future samples
of a species which appeared in the reference individual. The increasing prevalence
of multiple complete genomes per species leads to a desire for more than
one reference sequence per species. Older models in computational 
genomics do not have the power required to adequately handle multiple 
reference genomes. It is now seen as desirable to create a pan-genome, 
a single representation of all available gene sequences from a species, 
which can be viewed as a single entity. The SplitMEM algorithm is 
designed to take multiple genomic lines and convert them to a compressed
de Bruijn graph pan-genome representation, which will enable the 
isolation of common features in the genomes so that characteristics 
of the entire species or clade can be identified while gene sequences 
specific to an individual organism can be de-emphasized\cite{Marcus14}.

\section{Algorithm: Data Structures with Unpredictable Timing}
In most modern networks their is a stable, generally deterministic, data
structure in place to handle requests. By definition, these
deterministic systems are very predictable. Unfortunately, because of their
predictability, they can be abused. Users who understand the nature of 
the network can overload the system with a series of time consuming
requests, which eventually bog down the network. These denial-of-service
(D.O.S.) attacks can occur for a variety of reasons ranging from 
breaching security to gaining an advantage in competitive gaming. In 
general, D.O.S. attacks rely on knowing how the network will respond to 
every request. In theory, if an adversary could not predict a network's 
response, then they could not mount a D.O.S. attack on the network. 
Knowing this, we are motivated to 
create a data structure that has unpredictable timing 
for any given input, while still performing at a near optimal speed. 
Such a data structure must be able to perform the same operations as 
the original system, while also non-deterministically, i.e. randomly, 
altering itself. This new system should be able to resist all timing-
dependant attacks,
even from attackers who know the algorithm and the previous i/o values used.
In their paper, Darrell Bethea and Michael K. Reiter discuss a data
structure for set operations with these desirable characteristics\cite{Bethea09}.

\section{Algorithm: Quantum Factoring}
Factoring large  integers is a notoriously hard problem for deterministic 
computing machines. The best method we have now  for factoring is based 
on the probability of a given number being a factor of a larger number. 
Therefore, factoring is a problem which naturally lends itself to a
quantum computer. A quantum computer behaves like a computer with a 
random number generator integrated into each basic computing circuit.
This new aspect enables the computer to perform tasks which involve 
specific uncertainty in every step of computation much more easily.
There are some defining differences involved in quantum computation
which need novel design of the computing system. Referring to quantum
physics, we know that a particle can be in 
different positions at the same time with different probabilities. 
However, the total probabilities of being in all positions sum up to 1. 
Because of this, a quantum state representing some possible outputs
should have the sum of possibilities of being in all positions equal to
1. Thus only a unitary transformation is allowed in quantum computation.
This guarantees total probabilities equal to 1. Although there are no
real quantum computers at this point, it is useful to consider algorithms
to run on one, partly because they may be built in the future, but also
because we may learn by modelling those algorithms on current computing machines.
Peter Shor proposed an algorithm which takes advantage of the properties
of quantum computers to factor prime numbers in time polynomial to the
size of the integer to be factored\cite{Shor97}.

\pagebreak

\begin{thebibliography}{9}

\bibitem{Marcus14}
  Marcus, S., H.Lee, and M.C. Schatz,
  \emph{SplitMEM: a graphical algorithm for pangenome analysis with suffix skips.},
  Bioinformatics,
  2014. 30(24): p 3476-3483.
  
\bibitem{1000genomes}
	http://www.1000genomes.org.
	
\bibitem{1001genomes}
	http://www.1001genomes.org.
	
\bibitem{100KProject}
	http://100kgenome.vetmed.ucdavis.edu.
	
\bibitem{Haussler08}
	Haussler, D., et al.,
	\emph{Genome 10K: a proposal to obtain whole-genome sequences for 10,000 vertebrate species.}
	The Journal of Heredity,
	2008. 1000(6): p 659-674.
	
\bibitem{Shor97}
	Shor, Peter W.,
	\emph{Polynomial-time Algorithms for Prime Factorization and Discrete Logarithms on a Quantum Computer.}
	SIAM Journal on Computing,
	1997. 26(5), p 1484-1510.
	
\bibitem{Bethea09}
	Bethea, Darrell, and Michael K. Reiter,
	\emph{Data Structures with Unpredictable Timing}
	ESORICS,
	2009.
	
\end{thebibliography}

\end{document}

