\documentclass[11pt]{article}
\usepackage[margin=1in]{geometry}
\usepackage{algorithm}
\usepackage{algpseudocode}

%opening
\title{CSCI 432, Group Project Part 3}
\author{Group 10}

\begin{document}

\maketitle

%Intro

	In most modern networks their is a stable, generally deterministic, data
structure in place to handle requests, which is, by definition, very predictable. As with many systems, this was designed with usability, rather than security, in mind. However, users with malicious intent, such as those trying to gain an advantage in online gaming or those wanting breach security for other reasons, have shown that this predictability is also a vulnerability. They use the predictability of the data structure to send overwhelming requests to the network in attacks known as denial-of-service, or D.O.S. attacks. If this predictability is taken away and the ability to know how a network will respond to every request is eliminated, a D.O.S. attack would be difficult or impossible to mount. This is where our particular algorithm comes in. The purpose of the algorithm is to create a data structure that has unpredictable time for any input, while still performing at near optimal speed. Such a data structure must be able to perform the same operations as 
the original system, while also non-deterministically, i.e. randomly, 
altering itself. This new system should be able to resist all timing-
dependent attacks,
even from attackers who know the algorithm and the previous i/o values used.
In their paper, Darrell Bethea and Michael K. Reiter discuss a data
structure for set operations with these desirable characteristics\cite{Bethea09}.

%Description

	The problem of creating a data structure with unpredictable timing has been 
attempted previously and in general there are two types of structures.
There are structures whose goal is to make the worst case unpredictable and 
those whose goal is to make the worst case the least detrimental. Both types 
have their limitations and can still be abused by a cleaver enough foe.
A algorithm creating truly unpredictable timing must change its internal 
structure often, if not every iteration, so that its timing cannot be predicted, particularly based upon previous iterations. 
To accomplish these goals this algorithm uses a unique implementation of skip 
lists along with a dynamic origin to create a data structure with unpredictable timing.

	At the core of this algorithm are skip lists, which are specialized linked lists. 
Skip lists are a series of linked lists $L_{0}$ through $L_{m}$, where m is not constant, meaning the number of linked lists can change depending on the input. The base list $L_{0}$ contains every value entered into the set, $L_{1}$ will contain half the values of $L_{0}$, and so on until $L_{m}$ only contains a single value. What's important about these lists is that for any list $L_{k}$ every value on list $L_{k}$ is also on list $L_{k-1}$. This fact, along with the lists being sorted, makes traversing through the lists extremely efficient. To traverse the 
lists we start at list $L_{m}$ and move across our values until the next value is greater 
than our search value. Then we go down a list to list $L_{m-1}$ and search that tree starting 
at the value we stopped with on tree $L_{m}$. 

	The next step of the algorithm is to take the normal implementation of skip 
lists and alter it slightly. In the initial implementation of skip lists if you
get to the greatest value without having found your search value, then your search value must not 
be in the list. This method assumes that we start at with the smallest value in 
the lists and move to the largest but this is predictable. This algorithm 
needs to be able to start at any point, including the origin, and check all values in the 
lists regardless of whether they are larger or smaller than the starting value. To do this the creators of this 
algorithm made the skip lists circular, meaning that rather than the list having a specific head or tail the largest value points to the smallest value. Only when traversing past the starting point a second time do we stop.
  
	The last critical step to making this algorithm truly timing unpredictable is giving it a 
dynamic origin. Ultimately the goal is to choose a new, semi randomly chosen 
origin after every function call, loopup(), insert(), or delete(), is made. This 
makes it so no matter how many trials or repetitions of the same value an attacker 
attempted he or she could never guess where the origin was. In the algorithm this is accomplished 
by adding a few variables to the node class such as "skipped" and "index" which are maintained, 
in a timely fashion, throughout the function calls. First a node is randomly selected from 
all nodes, then using a calculation with the "skipped" and "index" variables from the 
randomly selected node a new origin node is selected. This calculation is quite in depth 
and can go as far as to change the entire height of the tree during when performed. 

%Conclusion

	To summarize, predictability in request handling, although an efficient solution, makes the network vulnerable to certain attacks. In an effort to prevent such attacks, this algorithm creates a data structure with unpredictable timing which eliminates the effectiveness of D.O.S. attacks while maintaining an optimal speed and original functionality.

%Algorithm

The set ADT supports three basic operations, lookup, insert, and delete. We provide pseudocode
for those operations here.

\begin{algorithm}
\caption{TUFL-LOOKUP}
\textbf{Input:} search element v

\textbf{Output:} v if v is found, null otherwise

\begin{algorithmic}[1]
\State $cur \gets head$
\State $level \gets head.height$
\State $done \gets false$
\While{!done}
	\If{cur.next \textless \ v \ or \ v\  \textless\  head\  \textless\  cur.next}
		\State $cur \gets cur.next$
	\ElsIf{cur.next = v}
		\State $head \gets$\ \Call{CHOOSE-NEW-HEAD}{head}
		\State \Return v
	\ElsIf{level \textgreater\ 1}
		\State $level \gets level-1$
	\Else
		\State $done \gets true$
	\EndIf
\EndWhile
\State $head \gets$\ \Call{CHOOSE-NEW-HEAD}{head}
\State \Return null
\end{algorithmic}
\end{algorithm}

\begin{algorithm}
\caption{TUFL-INSERT}
\textbf{Input:} new element v

\textbf{Output:} True if inserted, false if exists

\begin{algorithmic}[1]
\State $height \gets$\ \Call{CHOOSE-RANDOM-HEIGHT}{}
\State $location \gets$\ \Call{FIND-INSERT-LOCATION}{v}
\If{v = location}
	\State $head \gets$\ \Call{CHOOSE-NEW-HEAD}{height}
	\State\Return false
\Else
\State $head \gets$\ \Call{CHOOSE-NEW-HEAD}{}
\State\Call{ADD-ELEMENT}{location, v, height}
\State\Return True
\EndIf
\end{algorithmic}
\end{algorithm}



\begin{algorithm}
\caption{TUFL Delete}
\textbf{Input:} element to delete v

\textbf{Output:} true if v was removed, false if it did not exist

\begin{algorithmic}[1]
\State $location \gets$\ \Call{FIND-INSERT-LOCATION}{v}
\If{v = location}
	\State \Call{DELETE-ELEMENT}{location}
	\State $head \gets$\ \Call{CHOOSE-NEW-HEAD}{}
	\State \Return True
\Else
	\State $head \gets$\ \Call{CHOOSE-NEW-HEAD}{}
	\State \Return False
\EndIf
\end{algorithmic}
\end{algorithm}

\pagebreak

\begin{thebibliography}{9}
	
\bibitem{Bethea09}
	Bethea, Darrell, and Michael K. Reiter,
	\emph{Data Structures with Unpredictable Timing}
	ESORICS,
	2009.
	
\end{thebibliography}

\end{document}


