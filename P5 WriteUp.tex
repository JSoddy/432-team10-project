\documentclass[11pt,twocolumn]{article}
\begin{document}{11pt}

So far our work has been focused on two parts of our project, Skip List implementation and 
generating unique data sets to test our program with. Implementing skip lists involved 
creating a data structure that is linked not only from left to right but also up and down. 
It also caused us to make unique add, search, and delete functions that are different from 
normal link list functions. The data sets we have created will help us illustrate the 
differences in how our algorithm will handle an attack versus a normal data structure. 
They will also ensure that our data structure is not gimmicky in the sense that it only 
performs better on very specific data.

Now that we have the normal skip list implementation complete our next goals are to make 
the Skip list circular and implementing a randomized starting node for our lists. Both of 
these should be fairly easy to accomplish with our current skip list implementation. Making 
the lists circular will involve connecting the starting and ending nodes and also modifying 
our search, delete, and add functions so that they account for an unknown starting point. 
Creating a randomized starting node will involve decisions about how the data structure 
should be modeled after the change. Once we have our data structure complete we can move 
to the video production of our project. Our video will begin by explaining the purpose of 
our data structure and explaining the essentials of how it functions. Then we will show, 
and explain, a series of illustrations that show the differences in performance of our 
algorithm versus other common algorithms when under an attack. This will likely involve 
programming the basic form of some common data structures which should not be too 
difficult.

The major difficulties we have faced throughout this project are the standard issues every 
group runs into. First is finding a common time to meet and work on the project, everyone 
has unique schedules that don't always coincide with each other. Secondly is finding a 
part of the project that everyone will excel at doing, each of us have a unique skill set 
that might be better served performing one task over another. Lastly is getting the 
programs used to work for everyone in the group. While using version control has 
undoubtedly helped keep everyone on the same page throughout working on our project 
setting up git-hub and learning latex has been taxing at times.

\end{document}