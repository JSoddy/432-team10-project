\documentclass[11pt,twocolumn]{article}
\usepackage[margin=1in]{geometry}

\usepackage{algorithm}
\usepackage{algorithmicx}
\usepackage{algpseudocode}

%opening
\title{Motivating Data Structures With Unpredictable Timing}
\author{Group 10\\
\small{Tao Huang, Travis Wentz, James Corbett and James Soddy}}

\begin{document}

\maketitle

Protection from malicious attackers is easily understood as a general motivation
for creating data structures with unpredictable timing. More difficult to understand,
however, are the precise ways in which such structures are able to prevent adversaries
from exploiting a system. Our intention is to clearly illustrate, through clear,
high level descriptions and illustrated examples, precisely the way in which attacks
can compromise standard systems, and how the protection offered by randomized
data structures works.

\section{Motivation}
What is the problem that this algorithm is trying to solve (note: you may want 
to borrow from your P-2 write-up).

\section{Algorithm}
Brief description of the main points of the algorithm.

\section{Analysis Problems}
These data structures have been analyzed, and details of their suitability and
efficiency have been given in the Bethea/Reiter paper\cite{Bethea09}. However,
making use of this information requires wading through material such as:

% Insert crazy equation here.

It is our intention to present the details of the algorithm in such a way that
it is easily understood by anyone with a basic grasp of computing principles,
or even a motivated layman.


\section{Discussion}
Conclude with a discussion.  Things that you might want to consider to put in 
the discussion (or maybe in their own sections) include: What are the biggest 
challenges that you will face for the remainder of this project?  

\bibliographystyle{plain}
\bibliography{bibfilename} 

\newpage
\onecolumn
\appendix
\section{Timeline}
(NOTE: it is good form to have a reference to this bibliography somewhere in 
the main body of your paper).

The following are a list of tasks that we need to accomplish in order to 
complete our project:

\paragraph{Structure the video.} We need to plan a detailed outline of our 
video.  We will meet together to do this.

\paragraph{Decide materials needed.}  Make sure you have everything you need in 
order to make this 4-5 minute video!

... and so forth.

The following table gives the timeline of how we plan to accomplish these tasks:

\begin{table}[h!]
\centering
\begin{tabular}{ |l | c | r|}
  \hline
  Date & Who? & Short Description \\
  \hline
  \hline
  10/30 & All & Structure the video. \\
  \hline
  10/31 & Joe S. & Decide materials needed. \\
  \hline
  \ldots & \ldots & \ldots \\
  \hline
  11/29 & All & Final review of video. \\
  \hline
\end{tabular}
\end{table}

\pagebreak

\begin{thebibliography}{9}
	
\bibitem{Bethea09}
	Bethea, Darrell, and Michael K. Reiter,
	\emph{Data Structures with Unpredictable Timing}
	ESORICS,
	2009.
	
\end{thebibliography}

\end{document}
