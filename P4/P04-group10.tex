\documentclass[11pt,twocolumn]{article}
\usepackage[margin=1in]{geometry}

\usepackage{algorithm}
\usepackage{algorithmicx}
\usepackage{algpseudocode}

%opening
\title{Motivating Data Structures With Unpredictable Timing}
\author{Group 10\\
\small{Tao Huang, Travis Wentz, James Corbett and James Soddy}}

\begin{document}

\maketitle



\section{Motivation}
What is the problem that this algorithm is trying to solve (note: you may want 
to borrow from your P-2 write-up).

\section{Algorithm}
Brief description of the main points of the algorithm.

\section{Analysis}
Optional: you are welcome to provide a brief proof of correctness or a 
running time analysis of the algorithm.  The difficulty of doing this will, of 
ocurse, depend on your algorithm.

\section{Discussion}
Conclude with a discussion.  Things that you might want to consider to put in 
the discussion (or maybe in their own sections) include: What are the biggest 
challenges that you will face for the remainder of this project?  

\bibliographystyle{plain}
\bibliography{bibfilename} 

\newpage
\appendix
\section{Timeline}
(NOTE: it is good form to have a reference to this bibliography somewhere in 
the main body of your paper).

The following are a list of tasks that we need to accomplish in order to 
complete our project:

\paragraph{Structure the video.} We need to plan a detailed outline of our 
video.  We will meet together to do this.

\paragraph{Decide materials needed.}  Make sure you have everything you need in 
order to make this 4-5 minute video!

... and so forth.

The following table gives the timeline of how we plan to accomplish these tasks:

\begin{table}[h!]
\centering
\begin{tabular}{ |l | c | r|}
  \hline
  Date & Who? & Short Description \\
  \hline
  \hline
  10/30 & All & Structure the video. \\
  \hline
  10/31 & Joe S. & Decide materials needed. \\
  \hline
  \ldots & \ldots & \ldots \\
  \hline
  11/29 & All & Final review of video. \\
  \hline
\end{tabular}
\end{table}
\end{document}
