\documentclass[11pt,twocolumn]{article}
\usepackage[margin=1in]{geometry}

\usepackage{amsmath}

%opening
\title{CSCI 432 Group Project, Progress Report}
\author{Group 10\\
\small{James Corbett, Tao Huang, James Soddy, and Travis Wentz}}

\begin{document}

\maketitle

We are working on our project to implement and explain the randomized
set data structure set out by Bethea and Reiter in their 2009 
paper\cite{Bethea09}. Our goal is to use visual information as well
as real numbers collected from our implementations to show exactly
how a set data structure can be exploited, and how the randomized
implementation can protect from those attacks.

So far our work has been focused on two parts of our project, skip list implementation and 
generating data sets which we can use to test our program and illustrate the relative
strengths and weaknesses of randomized implementations. Implementing skip lists involved 
creating a data structure that is linked not only from left to right but also up and down. 
It also required us to make unique add, search, and delete functions that are different from 
normal link list functions. The data sets we have created will help us illustrate the 
differences in how our algorithm will handle an attack versus a normal data structure. 
They will also ensure that our data structure is not gimmicky in the sense that it only 
performs well on specific data.

Now that we have the normal skip list implementation complete, we are working on adapting our
code for the randomized skip list set implementation. This will require us to make the skip 
list circular and implement randomization of the starting node and 'height' of elements in our
lists. Given our current skip list implementation, we don't foresee any difficulty in the adaptation. 
Making the lists circular will involve connecting the starting and ending nodes and also modifying 
our search, delete, and add functions so that they account for an unknown starting point. 
To create the randomized starting node will simply involve inserting a new node as the head,
and adjusting its height appropriately.

Once we have our data structure complete we can move 
to the video production of our project. Our video will begin by explaining the purpose of 
our data structure and explaining the essentials of how it functions. Then we will show, 
and explain, a series of illustrations that show the differences in performance of our 
algorithm versus other common algorithms when under an attack. 

The major difficulties we have faced throughout this project are the standard issues every 
group runs into. First is finding a common time to meet and work on the project, everyone 
has unique schedules that don't always coincide with each other. Secondly is finding
parts of the project at which each member will excel. Each of us has a unique skill set 
that might be better utilized performing one task over another. Last is getting the 
tools we are using to work well for everyone in the group. While using version control has 
undoubtedly helped keep everyone on the same page throughout our project, 
setting up git-hub and learning latex has been taxing at times.

Although we have fallen somewhat behind the schedule we laid out in our time line,
we are not overly concerned about our ability to complete the project and have a quality
presentation. We have thoroughly planned our project, and prepared the tools we will
need to complete it. From here our main priority is finding suitable blocks of time that
we will all be able to work together to complete the work that requires all of us.

\pagebreak
\onecolumn

\begin{thebibliography}{9}
	
\bibitem{Bethea09}
	Bethea, Darrell, and Michael K. Reiter,
	\emph{Data Structures with Unpredictable Timing}
	ESORICS,
	2009.
	
\end{thebibliography}

\end{document}